%This is an example for a chapter, additional chapter can be added in the skeleton-thesis
%To generate the final document run latex, build and quick build commands on the skeleton-thesis file not this one.
\chapter{Chapter One Title Here} \label{chap:ref1}

Chapter one text~\cite{article1,book1}. 
\begin{align}
  F &= m a \\
    &= \frac{dp}{dt} \label{eq:ref1}
\end{align}
See Appendix~\ref{app:ref1}.


% for glossary usage:
%   \newacronym will be defined the first time it is used
%   \newglossary will not be defined in the text, but it appears in the Abbrev. section, if you have one.
%
A glossary acronym like \gls{tla} is defined the first time, and found in the Abbreviations section, see page~\pageref{sec:abbrevs}. While the nomenclature is not defined automatically it is still found in the Abbreviations section, like \gls{masssun}. Let's go back to the \gls{tla} for more on this story.


%%%
%%%
%%%
%%%
%%%
\section{Examples}

\subsection{An Equation}
Given
\begin{equation}  \label{eq:ex}
   f  = x / t ,
\end{equation}
where $f$, $x$, and $t$ are undefined.

\subsection{A Table}

\subsubsection{Example One}
With latex a usual table usual might look like Table~\ref{tab:example1}.
% 
\begin{table} [htbp] %[htbp]
  \centering \ssp
  \caption[Caption text in TOC]{Table caption text. }
  \label{tab:example1} 
  \begin{tabular}{ccc}
  \hline  \hline   
   Column  &  Column    &  Other     \\ %
   {[km]}  &   {[s]}    &            \\ %   
  \hline   
    50.    &  $ 0.12 \pm 0.07 $  &  $1.24\times 10^{-2} $  \\  
    40.    &  $ 2.12 \pm 0.16 $  &  $2.16\times 10^{-2} $  \\  
    30.    &  $ 1.6  \pm 0.52 $  &  $3.34\times 10^{-2} $  \\  
    20.    &  $ 9.62 \pm 0.16 $  &  $7.2 \times 10^{-2} $  \\  
    10.5   &  $12.10 \pm 0.22 $  &  $1.8 \times 10^{-1} $  \\  
  \hline   \hline  
  \end{tabular}
\end{table}



\subsubsection{Example Two}

The package \texttt{siunitx} helps numerical (e.g.~\num{1e4}, \num{2.4(1)}), unit (e.g.~\si{\per\micro\ampere\per\nano\second}, \si{\molecules/\cm\squared}), and combined formatting (e.g.~\SI{4.0(4)}{\km\per\hour}). Also for table formatting by vertical alignment (see Table~\ref{tab:example2}).  

\begin{table} [bhp] %[htbp]
  \centering \ssp
  \caption{Caption text using optional \texttt{siunitx} package and \texttt{booktab} package. Also here is a long caption that shows this is a single spaced environment.}
  \label{tab:example2} 
  \begin{tabular}{S[table-format=2.1]  
                  S[table-format=2.2(2)]    
                  S[table-format=1.2e2]       } 
  \toprule    
  {Column } & {Column}     &  {Other}   \\ % siunitx expects a number
  {[km]}    & {[s]}        &            \\ %  unless '{}' are used  
  \midrule %
    50.     &   0.12( 7)   &  1.24e-2   \\  
    40.     &   2.12(16)   &  2.16e-2   \\  
    30.     &   1.6 (52)   &  3.34e-2   \\  
    20.     &   9.62(16)   &  7.2 e-2   \\  
    10.5    &  12.10(22)   &  1.8 e-1   \\  
   \bottomrule 
  \end{tabular}
\end{table}